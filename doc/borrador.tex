\hypertarget{uxedndice}{%
\section{Índice}\label{uxedndice}}

\begin{enumerate}
\def\labelenumi{\arabic{enumi}.}
\tightlist
\item
  Temática
\item
  Referentes
\item
  Enfoque
\item
  Planteamiento Metodológico
\item
  Planteamiento Teórico
\item
  Conclusiones
\item
  Bibliografía
\end{enumerate}

\hypertarget{i-temuxe1tica}{%
\section{I: Temática}\label{i-temuxe1tica}}

El tema principal de este proyecto, si bien creo que ha ido madurando y
evolucionando a través del propio proceso de creación, y, de algún modo,
emergiendo y desarrollándose orgánicamente, gira alrededor de la
fantasía, la creación, y el desarrollo de un mundo personal.

El enfoque de esta temática central, que he ido manteniendo desde las
primeras obras, ha sufrido, sin embargo, un cambio drástico: de piezas
más bien ilustrativas, que comunicaban su intencionalidad apoyándose
principalmente en el discurso a su alrededor y una simbología críptica,
han surgido obras, las más tardías, que no se centran tanto en el
discurso, sino en las metáforas visuales, o incluso en la sensación que
transmiten. También, el apoyo teórico y la motivación del proyecto ha
dado un giro importante desde el inicio: al principio, quería crear un
mundo para crear situaciones con las que comunicar una intencionalidad,
sobre todo política; pero he descubierto que el mundo ya estaba creado
en su mayor parte: no ha sido necesario un diseño pormenorizado de
situaciones ni conceptos extraños, sino más bien una búsqueda de
sensaciones e imágenes fantásticas, que no surgían de una reflexión
pragmática sino de una emotiva. Es, por tanto, que considero que aquello
que quería que fuese un lugar perfectamente definido, se ha tornado en
algo nebuloso e indeterminado, que sólo toma cohesión y se presenta como
lugar cuando se observa por partes.

De la propia reflexión sobre la nebulosa contemplación del mundo ha
surgido una metareflexión sobre el propio proceso generativo y casi
errático de la creación de este mundo, y de la naturaleza de la propia
obra: ¿es esta el mundo creado? ¿o es más bien el proceso agregado de
creación, con todos los cambios, las ideas incompletas, la evolución de
las aceptadas y la propia incompletitud e indefinición de muchas partes
del mismo?

\begin{center}\rule{0.5\linewidth}{0.5pt}\end{center}

El por qué de la selección de este tema es ha ido también fluctuando con
el tiempo, aunque algo menos. Desde muy pequeño me han fascinado todo
tipo de historias y me considero un ávido lector. También desde pequeño
he tenido claro que aquello que más me satisfacía era la creación, en el
sentido amplio de la palabra: las construcciones, dibujar, inventar
juegos, y, algo más tarde, la robótica, la programáción, la escritura,
la música o la pintura, han sido algunas de mis aficiones. Aunque no
tengo muy claro la razón, las historias de fantasía son las que más me
han fascinado siempre, pasando por el floklore europeo (la mitología
celta y galesa, la grecorromana, las historias del Rey Arturo, las
leyendas y romances castellanos), la fantasía juvenil (como las obras de
Laura Gallego, Brandon Sanderson, Patrick Rothfuss o George R.R.
Martin), la fantasía mas ``académica'' o literaria (como puede ser la
obra de J.R.R Tolkien, Ana María Matute o Neil Gaiman), o los
videojuegos. De estas historias, lo que más me atrae siempre es la
existencia de mundos a parte de este, de una manera escapista: la
curiosidad y la necesidad de explorar lo desconocido y de descubrir que
no todo es como parece. Esta sensación viene del sentimiento (erróneo),
de haber perdido la capacidad de asombro ante el mundo, pues está todo
descubierto.

Así, en resumidas cuentas, el tema de este proyecto es \textbf{la
creación y el descubrimiento orgánicos de un mundo fantástico y propio,
que emerge de manera automática a partir de un imaginario desarrollado
pasiva y activamente a lo largo de toda mi vida}.

\hypertarget{referentes}{%
\section{Referentes}\label{referentes}}

Los referentes que he tomado para este proyecto no están escogidos
\emph{ad hoc} para el proyecto, sino que son los autores y los
materiales que consumo habitualmente y de los que bebo en la mayoría de
mis obras, así como en otros aspectos de mi vida.

\hypertarget{literatura}{%
\subsection{Literatura}\label{literatura}}

En el campo de la literatura, que es en el que más informado y
experimentado me considero, mis principales fuentes de inspiracion son:

\hypertarget{j.r.r.-tolkien}{%
\paragraph{J.R.R. Tolkien}\label{j.r.r.-tolkien}}

Tolkien es uno de mis escritores favoritos, y uno de los creadores que
más admiro. Es indiscutiblemente el padre de la fantasía moderna, y mi
referente principal en el ámbito más minucioso y quasi-científico de la
creación de mundos, el \emph{worldbuilding}. {[}BIB{]} De sus obras,
\emph{El Silmarillion} y \emph{El Señor de los Anillos} son verdaderos
tratados y obras cumbres de la creación de un mundo y de cómo infundirle
vida. \emph{El Hobbit} es un cuento que a primera vista parece menor a
los anteriores, pero considero que es una obra maestra porque alcanza,
con unas pinceladas muy bien dadas, crear las sensaciones de ese mundo,
sin la necesidad de pormenorizarlo todo.

\hypertarget{brandon-sanderson}{%
\paragraph{Brandon Sanderson}\label{brandon-sanderson}}

De los escritores de fantasía en el mercado actual, es mi favorito, y es
un modelo a seguir en constancia y tesón. Lo que más interesante
encuentro en este autor, más allá de su maestría con el worldbuilding
que rivaliza con la de Tolkien, es el énfasis que pone en la creación y
definición de sistemas de reglas lógicos por los que se rigen sus
mundos, que les dan una entidad realista a la par que extraña. También
es muy interesante su \emph{Cosmere} que es el universo en el que se
desarrollan sus historias, en el que todo está interrelacionado.
{[}BIB{]} \emph{El Archivo de las Tormentas} es su obra magna, aunque
todos sus libros son muy buenos.

\hypertarget{neil-gaiman}{%
\paragraph{Neil Gaiman}\label{neil-gaiman}}

Neil Gaiman es un escritor que me parece muy interesante, sobre todo por
la maestría con la que crea sus personajes e imbuye sus mundos de magia
y misterio sin necesidad de ahondar en detalles. {[}BIB{]} En
\emph{Stardust} encontramos referencias a antiguas leyendas británicas,
así como el concepto de \textbf{Otro Mundo} que está tan solo al otro
lado de una tapia.

\hypertarget{ana-maruxeda-matute}{%
\paragraph{Ana María Matute}\label{ana-maruxeda-matute}}

De Ana Matía Matute me fascinó \emph{Olvidado Rey Gudú}, en la que el
mundo se comporta como otro más de los personajes, fluctuante,
indefinido, y acaba por no haber existido nunca, lo que, en mi opinión,
le da una capa más de fantasía.

Otros escritores en los que quizá no me inspiro tanto, pero a los que he
leído y cuyas obras me han influido de alguna otra manera, son por
ejemplo Andrewj Sapowski, con su \emph{Saga del Brujo}, Shakespeare,
sobre todo con obras como \emph{El sueño de una noche de Verano} o
\emph{Romeo y Julieta}, Gabriel García Márquez, con \emph{Cien años de
Soledad} y el Realismo Mágico, William Golding con \emph{La Princesa
Prometida}, o Ursula K. Le Guin, con su saga \emph{Un brujo de
Terramar}. También autores más clásicos, como Dante o Bocaccio me
parecen muy interesantes por su forma de tratar el imaginario legendario
y místico de la Edad Media.

\hypertarget{artes-pluxe1sticas}{%
\subsection{Artes Plásticas}\label{artes-pluxe1sticas}}

Dentro de las artes plásticas admiro a casi todos los autores más
conocidos, aunque si que creo que en este campo no soy tan ducho como en
el anterior. De algunos autores me interesa mucho su discurso y su
mensaje, pero también hay muchos otros que me interesan por las técnicas
que emplean.

\hypertarget{turner}{%
\paragraph{Turner}\label{turner}}

De Turner me interesa mucho la pseudo abstración y la forma de la que
emplea el color, anticipándose, en mi opinión, al final del siglo XIX e
incluso a la Vanguardia. También me atrae mucho la sensación que
transmite su obra, de frenetismo y desasosiego, sin tener que ahondar en
detalles a la hora de pintar.

\hypertarget{david-kaspar-friederich}{%
\paragraph{David Kaspar Friederich}\label{david-kaspar-friederich}}

De Kaspar Friederich, aunque casi toda su obra de paisajes me parece
magistral, me interesa sobre todo su tratamiento de las ruinas, que
siempre están ahí para recordar, evocando tiempos pasados y fusionándose
con la naturaleza.

\hypertarget{el-bosco}{%
\paragraph{El Bosco}\label{el-bosco}}

De el bosco me fascina la soltura con la que crea situaciones y escenas
totalmente fantásticas y alocadas, casi surrealistas, además del
contexto histórico en el que lo hace. También me interesa su uso del
color, muy profuso y saturado, y, a veces, me parece hasta cómico.

\hypertarget{goya}{%
\paragraph{Goya}\label{goya}}

De Goya me interesa sobre todo la gestualidad y la forma que tiene de
crear imágenes con pinceladas muy sueltas, pero que crean el efecto
correcto a cierta distancia. Sus pinturas negras son una referencia
importante para algunos obras en tinta.

\begin{center}\rule{0.5\linewidth}{0.5pt}\end{center}

Otros autores que quiero destacar, aunque quizá no hayan sido tan
influyentes en este proyecto pero que, sin duda, lo son para mi día a
día, son por ejemplo Leonardo Da Vinci y Durero, de los que me interesa
la genialidad y el aspecto polifacético del Renacimiento, así como su
obra más teórica sobre la pintura y sobre los propios materiales
pictóricos, que es otro tema que me interesa muchísimo. En el campo de
las técnicas y materiales tradicionales mi fuente antigua de referencia
es \emph{El Libro del Arte}, de Ceninno Ceninni, autor del que se sabe
poco hoy en día. También cabe destacar en este ámbito a Ralph Meyer,
aunque su tratado sobre materiales artísticos me parece menos
interesante por su contemporaneidad. Velázquez, Rubens, Tiziano y
Tintoretto también están entre mis pintores favoritos, aunque y me
gustan sobre todo sus obras de mitología. También me interesa mucho
William Blake, aunque no conozco mucho de su obra.

\hypertarget{muxfasica}{%
\subsection{Música}\label{muxfasica}}

Dentro de la música, mis referentes principales, como autores, son sobre
todo Bach y Chopin, pero también me inspiran e interesan mucho la música
anónima de la Edad Media, las baladas sobre el Rey Arturo, o los
romances del Romancero Viejo. Ejemplos de estas podrían ser
\emph{Scarborough Fair}, {[}TODO: Buscar Balada del Rey Arturo{]}, o el
\emph{Romance del Conde Olinos}. Debo destacar también algunas bandas
sonoras, como la de Howard Shore para \emph{El Señor de los Anillos}, o
la de Koji Kondo para \emph{The Legend of Zelda}, que, aunque son
acompañamiento a obras que ya he mencionado, han jugado un papel muy
importante para mí.

\hypertarget{videojuegos}{%
\subsection{Videojuegos}\label{videojuegos}}

Los videojuegos han sido una parte muy importante de mi vida desde la
infancia, y creo no equivocarme al afirmar que muchos de ellos se pueden
considerar arte sin lugar a dudas.

\hypertarget{the-legend-of-zelda}{%
\paragraph{The Legend of Zelda}\label{the-legend-of-zelda}}

Es mi saga de videojuegos favorita, y posiblemente una de las
influencias más grandes de este proyecto. Dentro de los más de veinte
títulos que la componen, destacaría \emph{Skyward Sword} y \emph{Breath
of The Wild}, sobre todo por la sensación de explorar lo desconocido que
transmiten, de adentrarse en un mundo extraño que nunca antes has visto.
También es muy interesante \emph{Majora's Mask} por el mundo hostil y
desconocido que presenta, como un reflejo oscuro de Hyrule.

\hypertarget{minecraft}{%
\paragraph{Minecraft}\label{minecraft}}

Creo que \emph{Minecraft} es otro de los juegos que más me ha influido,
sobre todo por la libertad que otorga al jugador de hacer con el mundo
que se le presenta aquello que le plazca.

\begin{center}\rule{0.5\linewidth}{0.5pt}\end{center}

Otros videojuegos que me parecen interesantes y que me han influido,
aunque en menor medida, son, por ejemplo, \emph{Skyrim},
\emph{Civilization}, \emph{Final Fantasy IX} o \emph{The Witcher}.

\hypertarget{informuxe1tica}{%
\subsection{Informática}\label{informuxe1tica}}

La informática y todo aquello relacionado con la tecnología también es
algo que ha influido mucho en mí a lo largo de la vida y, en menor
medida, en este proyecto. Mis padres son los dos programadores, y yo
aprendí a programar con 14 años, por lo que creo que forma parte
integral de mí. Como referentes en este campo, incluyo a Alan Turing,
padre de la informática, y Ken Thompsom, que programó el sistema
opertaivo UNIX (del que Linux es el sucesor espiritual). Aunque a
primera vista puede parecer que la informática y la programación tienen
una relación tangencial con el arte, en la investigación para este
proyecto he encontrado inspiración en la cultura \emph{maker}, o en la
\emph{demoscene}, así como el la \emph{programación creativa}. Creo que
es muy interesante incluir estas disciplinas en la creación de un mundo,
pues permite que este se \emph{autogenere} {[}REPASAR{]}

\hypertarget{otras-influencias}{%
\subsection{Otras influencias}\label{otras-influencias}}

Otras influencias y referencias, importantes pero más generales son por
ejemplo todas las leyendas del folklore gaélico (galesas, irlandesas y
escocesas, así como gallegas y bretonas), sobre el Mundo de las Hadas --
\emph{Ávalon}, \emph{Anwwn} o \emph{Tír na nÓg}--, el concepto romano
equivalente de \emph{Orbis Alia}, y la mitología relacionada con este
mundo, así como sus personajes recurrentes (\emph{Arawn} u
\emph{Oberón}, rey de las Hadas, y \emph{Titania}, la reina); o las
historias de vampiros (tanto las góticas como \emph{Drácula}, o
contemporáneas como \emph{Entrevista con el Vampiro} o el \emph{Mundo de
Tinieblas}).

\hypertarget{enfoque}{%
\section{Enfoque}\label{enfoque}}

El enfoque del proyecto es lo que más ha cambiado desde su concepción
hasta el momento final. En un principio, la intecnción era criticar
ciertas conductas y situaciones que se dan en nuestro contexto actual a
través de la presentación del mundo como una suerte de parábola. Uno de
los temas que más me interesaba era el de la juventud, la preocupación
por el legado que nos dejarán las generaciones que ahora están a cargo
de todo, y la reivindicación de nuestra identidad propia y nuestros
derechos como personas con la misma legitimidad que gente más mayor. A
medida que el proyecto ha ido evolucionando me he dado cuenta de que lo
que en realidad he hecho ha sido hablar de mí: de mis sueños, mis
ideales, del mundo que me gustaría, de mi propia evolución personal y de
cómo se hha desarrollado mi imaginario personal desde la niñez hasta el
momento presente. Es un proyecto ciertamente intimista, que trata un
tema muy ligado con todos los aspectos de mi vida como lo es el afán y
la necesidad de crear, pero no para hacer reivindicaciones ni para
mostrarlo, sino porque lo hago de forma natural. En realidad, siento que
una de las reflexiones más importantes que he obtenido es la de que,
para mí y sobre mi trabajo, es mucho más importante, satisfactorio, e
incluso llegaría a decir que, donde reside la materia artística, es en
el proceso, en la creación, que sucede en el momento en el que se
proyecta, se piensa o se tiene una idea. Aunque considero importantes la
materialización y el desarrollo de esas ideas, pues muchas veces surgen
a partir de ellas otras nuevas, creo poder afirmar que la obra y lo
artístico, al menos en mi trabajo, está en la idea virtual, en el
pensamiento, intangible y nebuloso, y el lienzo o la realización de la
misma no es más que una manera de recordarla, comunicarla, y
registrarla.

\hypertarget{marco-teuxf3rico}{%
\section{Marco Teórico}\label{marco-teuxf3rico}}

Sobre la teoría involucrada en este proyecto, quiero hacer una
distinción clara entre dos conceptos que se desarrollan paralelamente en
el mismo, cuyos discursos teóricos difieren ampliamente en naturaleza.
El primero, la teoría, técnica y estrategias empleadas para la creación
de mundos, es relativamente sistemático y utilitarista, y trata, sobre
todo, de cómo se crean mundos fantásticos efectivos, apoyándose en los
propios métodos que usan escritores de renombre y en el consenso de la
comunidad especializada en torno a ello, y de cómo esto que proviene del
ámbito literario se puede adaptar al pictórico y plástico en general. El
segundo es el discurso que emana de mi propio proceso creativo, de mi
visión conjunta del mundo ficticio, más reflexivo y centrado en cómo se
percibe la creación del mundo como obra.

\hypertarget{de-la-creaciuxf3n-de-mundos}{%
\subsubsection{De la Creación de
Mundos}\label{de-la-creaciuxf3n-de-mundos}}

En el ámbito literario, específicamente en el de la literatura
fantástica y juvenil de los últimos años, ha surgido la tendencia de
seguir ciertas reglas, métodos y estrategias más o menos definidas, así
como diversos acercamientos, para conseguir envolver las obras en un
buen \emph{worldbuilding}, esto es, creación de mundos. Diversas
comunidades de \emph{worldbuilders} se esparcen por internet {[}TODO
REDDIT{]}, ciertas más generales, y algunas más especializadas. En pos
de la brevedad, listo a continuación algunas estrategias básicas y
acercamientos a la creación de mundos literaria: {[}TODO: Buscar
técnicas de WorldBuilding{]}

Cabe destacar además de estas técnicas, recursos muy útiles y de buenas
fuentes como {[}CLASES DE BRANDON SANDERSON, HISTORIA DE LA TIERRA
MEDIA{]}.

En cuanto a cómo se pueden aplicar algunos de estos métodos a la pintura
y a la plástica, queriendo que estas sean la obra principal, creo que es
sencillo si abstraemos la parte de creación del proceso y nos centramos
en la parte de realización y de cómo se enseña el mundo. Uno de los
preceptos casi unánimemente aceptados por los \emph{worldbuiders} es el
de evitar el exceso de información que se da al espectador (lector en el
caso literario) de golpe, sobre el mundo (conocido como
\emph{infodumping}), y esto es aplicable, incluso más, a la pintura: la
forma más efectiva de dar a conocer el mundo creado es conseguir
transmitir \textbf{las sensaciones que en él se experimentan}, crear una
atmósfera única y distintiva, en lugar de llenar la obra de referencias
y simbolismos crípticos que necesitan de una explicación compleja y un
contexto extenso sobre el mundo.

\hypertarget{del-proceso-de-creaciuxf3n}{%
\subsubsection{Del proceso de
creación}\label{del-proceso-de-creaciuxf3n}}

Una reflexión muy interesante que surge del proceso de la propia
creación del mundo y a la que ya he hecho referencia es la de cómo se
desarrolla y cómo se visualiza de una manera general y homogénea el
mundo, desde la visión del autor, que lo contempla todo como un
conjunto. Esto me ha llevado a desgranar una cierta sistemática y
cosmovisión alrededor del mundo, en constante cambio, que define las
bases que lo componen y su razón de ser, todo ello en relación a la
disposición del proyecto a medida que crecía, sobre una de las paredes
del taller.

Esta pared, que es, más allá que el espacio físico, la idea en conjunto
del proyecto, es el proceso de creación, y a la vez la muestra y ventana
del mundo creciente. Este mundo se da a conocer a través de ciertas
\emph{reglas sistemáticas}, \emph{imágenes sinestésicas}, sin procesar,
e \emph{ideas}, en diversos grados de maduración e importancia, tanto
visuales como escritas y abocetadas. Las reglas le dan \emph{coherencia}
al mundo, permitiendo, con una definición simple, la derivación sin
necesidad de intervención por parte del autor, una evolución orgánica.
Las imágenes sinestésicas le dan \emph{sensibiliad} y \emph{presencia},
una suerte de alma, que no es ni cómo se comporta, ni cómo es, sino
\emph{cómo se siente}. Las ideas le dan \emph{intención}, y
\emph{entidad}: más allá de su evolución elemental y las reglas por las
que se rige, que pueden ser incluso ignoradas por alguien profano sin
restarle valor al mundo, o cómo se sienten sus realidades y su
\emph{atmósfera}, son su carácter, cómo \emph{es} y cómo \emph{existe}.

Estas paredes o mapas mentales, son, por tanto, una instantánea agregada
de todo este conjunto, en el que todo se pliega y se solapa en torno al
concepto abstracto que el autor tiene en mente cuando piensa en su
creación.

\hypertarget{planteamiento-metodoluxf3gico}{%
\section{Planteamiento
Metodológico}\label{planteamiento-metodoluxf3gico}}

Apoyándome en las bases teóricas anteriormente expuetas, tanto las
propias como las adaptadas de la literatura, a lo largo del proyecto me
he ido centrando, sobre todo, en crear \emph{imágenes sinestésicas} e
\emph{ideas}. Aunque ha habido intentos de crear y desarrollar algunas
\emph{reglas sisetemáticas}, no he hallado una forma elegante de
incluirlas como obras por derecho propio y aisladas, sino que arropan y
dan contexto como bocetos, apuntes, y pruebas de concepto, superpuestas
y llenando los huecos entre obras e ideas de otra índole alrededor de
\emph{la pared} -- en verdad, ¿por qué no considerarlas como obras por
derecho propio, aunque expresadas de manera diferente? A lo que me
quiero referir es que no he conseguido o necesitado conferirles una
materialidad pictórica. De una u otra manera, creo que al quedar cada
obra confinada a un tipo de estos «bloques básicos» del mundo, el
resultado es un trabajo limpio, que se puede sostener por si mismo, pero
que de igual manera gana profundidad al rodearlo de los elementos de
\emph{la pared}.

Formalmente, y más que a partir de una decisión meditada, através de la
experimentación y la creación iterativa, he dado en formalizar como
paisajes «pseudo-abstractos» (esto es, que, sin tener una idea en mente
sobre lo que iba a pintar, los resultados se pueden interpretar como
paisajes) y algunas otras imágenes muy sueltas las \emph{imágenes
sinestésicas}, y como arquitecturas, algunos retratos y paisajes más
figurativos las \emph{ideas}. Esto se amolda razonablemente bien a la
propia definición de estos conceptos, pues a través de las
\emph{imágenes sinestésicas} pretendo transmitir sensaciones y una
atmósfera, y, através de las \emph{ideas}, la materialidad y realidad
del mundo.

\hypertarget{planteamiento-tuxe9cnico}{%
\section{Planteamiento técnico}\label{planteamiento-tuxe9cnico}}

La metodología de trabajo empleada para desarrollar el proyecto ha ido
cambiando a medida se producían los cambios en torno a temática, enfoque
e intencionalidad ya mencionados.

En los primeros trabajos, las ideass surgían más lentamente, y la
mayoría de las veces, sin reflexionar mucho, y tendía a no barajar más
ideas que la primera (centrando algunas veces demasiado esfuerzo para
hacerla encajar, aunque fuese forzosamente). A medida que progresaba, he
ido desarrollando una metodología de trabajo más sosegada, y, aunque no
muy centrada en la reflexión teórica o en la investigación, que han
surgido eminentemente \emph{a posteriori}, sí en la introspección y el
la observación de mi obra previa. Aunque, sin duda, muy guiado por la
intuición, defino este proceso como más sosegado porque se caracteriza
por la importacia de los bosquejos y bocetos previos al planteamiento de
la obra, que actúan como documento de reflexión tanto teórica como
plástica: en todos los proyectos, la idea evoluciona fundamentalmente
sobre el papel (o el propio código o el lenguaje), unos primeros esbozos
muy rápidos, bocetos de detalle y gradualmente más encauzados, hasta
llegar a pruebas de color y bocetos finales de línea.

\hypertarget{estrategias-conceptuales}{%
\subsection{Estrategias conceptuales}\label{estrategias-conceptuales}}

Las estrategias conceptuales a las que más partido les he sacado en este
proyecto han sido sobre todo los recursos de la retórica visual, como
metáforas y analogías, las indefiniciones -- dejar que el espectador
interprete las partes de la imagen y reconstruya la escena en base a su
percepción subjetiva, si bien no referido a la abstracción o la
interpretación de los conceptos, sí a la de los espacios, como las
perspectivas, o las direcciones.

\hypertarget{estrategias-pluxe1sticas}{%
\subsection{Estrategias plásticas}\label{estrategias-pluxe1sticas}}

Las estrategias plásticas que he ido empleando a través del proyecto han
sido fruto en su mayoría de un proceso de experimentación con el
material. La técnica que más he usado a lo largo del proyecto ha sido el
óleo: los efectos más interesantes que han surgido con este han sido las
texturas casi de acuarela que he conseguido con la superposición de
manchas muy diluidas y difuminadas. El temple es otra técnica con la que
he experimentado, aunque algo menos. Los usos más interesantes de este
me han resultado al mezclar el pigmento con muy poca yema, creando unas
texturas muy terrosas, casi de tiza, de colores muy vivos.

En cuanto al uso del color, a medida que el proyecto avanzaba, se hacía
más protagonista la gama cromática completa, muy superpuesta, y muchas
veces casi sin mezclar, creando así unas superficies irreales y
fantásticas como las que buscaba. También cabe destacar la importancia
de la gestualidad en algunas obras, casi abstractas, la incompletitud
deliberada de algunas partes, que deja patente el proceso, y el uso de
imprimaciones y materiales caseros.

\hypertarget{conclusiones}{%
\section{Conclusiones}\label{conclusiones}}

Observando el recorrido del proyecto, creo que tengo claro qué tipo de
obra y procedimientos funcionan bien: la simplicidad en la
representación es clave para crear obras rápidas y efectivas, y creo que
es la manera idónea para transmitir el cambio constante del mundo que
pretendo crear. Queda también patente que presentar este tipo de obra
como un conjunto, rodeada de bocetos, ideas, proyectos nacientes, todo
ello en un espacio que evoluciona, es un acierto y una propuesta
expositiva interesante.

La investigación y el tiempo invertido en este proyecto, ha dado sus
frutos, aunque ha habido momentos de frustración y de sequía de ideas.
Sobre todo ha sido un proceso de introspección muy interesante, que ha
suscitado reflexiones realmente importantes, ya no solo sobre el propio
proyecto, sino sobre mi proceso creativo, mi visión del arte, mis
motivaciones y mis intenciones, y creo que esto es realmente valioso.

La creación de este mundo es algo que no se acota solamente al presente
proyecto, sino que es algo que viene de lejos y que continuará iterando,
sobreescribiéndose y cambiando, difuso, durante mucho tiempo más.
También ha sido muy satisfactoria la creación de algunas de las obras
que lo componen, y la sensación de completitud al ir atando cabos entre
diferentes conceptos, y de conseguir hilar el propio discurso de una
manera coherente y clara.

En el periplo de búsqueda de inspiración para nuevos tipos de obra y de
representar el mundo, también me he topado con un tema muy interesante y
que llevaba buscando mucho tiempo: la solución al problema que se me
había planteado al intentar unir mi interés por la informática y la
programación, y las artes, la «programación creativa», o \emph{creative
coding}. Esta es una rama muy interesante tanto de la ingeniería
informática, como de las artes digitales y gráficas, que pretende crear
arte interactivo de manera procedural, matemática y programática.

En conclusión, este proyecto me deja con buen sabor de boca, y con la
seguridad de haber desarrollado un estilo y un lenguaje propios
relativamente maduro, así como el descubrimiento de nuevos campos para
explorar.

\hypertarget{anexo-exposiciuxf3n-del-proyecto}{%
\section{Anexo: Exposición del
proyecto}\label{anexo-exposiciuxf3n-del-proyecto}}

Este proyecto se presenta de dos maneras: en su formato físico sobre la
pared del taller, y de manera virtual en una suerte de videojuego. Todo
el código fuente para el último, así como una copia de este documento y
las licencias pertinentes se encuentran alojados en el siguiente
repositorio de \emph{GitHub}, junto con instrucciones para su correcta
instalación y experimentación.

Este formato intenta emular el espacio expositivo ideal en el que se
expondría este proyecto, además de aprovechar las capacidades del medio
para disponer multiplicar, redimensionar y contravenir las leyes
naturales en pos de conseguir crear un espacio irreal que asemeje a la
idea, difusa y abstracta, que existe del mundo construido en la mente
del autor.

\hypertarget{relaciuxf3n-de-obras}{%
\subsection{Relación de obras}\label{relaciuxf3n-de-obras}}
